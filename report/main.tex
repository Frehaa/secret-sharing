\documentclass[a4paper,oneside,12pt,final]{article} 
\usepackage[utf8]{inputenc} 

\usepackage{graphicx}
\usepackage{amsmath}
\usepackage{bm}
\usepackage{wrapfig}
\usepackage{float}
\usepackage[a4paper, marginparwidth=0pt]{geometry}
\usepackage{parskip}
\usepackage{lmodern}
\usepackage{pdfpages}
\usepackage{hyperref}
\usepackage{minted}
\usepackage{booktabs}
\usepackage{multirow}
\usepackage[bottom]{footmisc}

\usepackage{xstring}


\graphicspath{./img/} 
\setlength{\parskip}{\baselineskip}% 
\setlength{\parindent}{0pt}% 
 
\begin{document} 

\begin{titlepage}

\newcommand{\HRule}{\rule{\linewidth}{0.5mm}} % Defines a new command for the horizontal lines, change thickness here

\center % Center everything on the page

%-------------------------------
%	HEADING SECTIONS
%-------------------------------

\textsc{\LARGE IT University of Copenhagen}\\[1.5cm] % Name of your university/college
\textsc{\Large MAJOR HEADING}\\[0.5cm] % Major heading such as course name
\textsc{\large COURSE TITLE}\\[0.5cm] % Minor heading such as course title

%-------------------------------
%	TITLE SECTION
%-------------------------------

\HRule \\[0.4cm]
{ \huge \bfseries TITLE}\\[0.4cm] % Title of your document
\HRule \\[1.5cm]

%-------------------------------
%	AUTHOR SECTION
%-------------------------------

\begin{minipage}{0.4\textwidth}
\begin{flushleft} \large
% \emph{Authors:}\\
%John \textsc{Smith} % Your name
Frederik \textsc{Haagensen} <haag@itu.dk>
\end{flushleft}
\end{minipage}

% ~
% \begin{minipage}{0.4\textwidth}
% \begin{flushright} \large
% \emph{Supervisor:} \\
% %Dr. James \textsc{Smith} % Supervisor's Name
% Leon \textsc{Derczynski} <leod@itu.dk>
% \end{flushright}
% \end{minipage}\\[4cm]

% If you don't want a supervisor, uncomment the two lines below and remove the section above
%\Large \emph{Author:}\\
%John \textsc{Smith}\\[3cm] % Your name

%-------------------------------
%	DATE SECTION
%-------------------------------

%{\large 23. maj 2017}\\[1cm] % Date, change the \today to a set date if you want to be precise
{\today}\\[1cm] % Date, change the \today to a set date if you want to be precise

%-------------------------------
%	LOGO SECTION
%-------------------------------

\includegraphics[width=10cm]{img/logo.jpg}\\[1cm] % Include a department/university logo - this will require the graphicx package

%----------------------------------------------------------------------------------------

\vfill % Fill the rest of the page with whitespace

\end{titlepage}

\newpage

\section{Introduction} 

The project is a simple interactive console application, which lets the user
simulate the parties of a verifiable secret sharing protocol for educational
purposes. The user can create a new secret sharing session, simulate parties
reconstructing the secret, simulate a party verifying their share, and change a
party's share to see how the results of the former commands would change.

\begin{figure}[h]
\label{fig:help}
\center
\includegraphics[width=0.8\textwidth]{img/help-ss.png}
\caption{The applications help message which displays the available commands.}
\end{figure}

When initializing, the user can specify the secret, the number of parties needed
to reconstruct, the number of parties participating, as well as the random seed
used by the protocol.

\begin{figure}[h]
\label{fig:initialize-secret-sharing}
\center
\includegraphics[width=0.8\textwidth]{img/initialize-print-ss.png}
\caption{A new secret sharing session established with the polynomial created, 
         the generator $g$, the party's shares, and the commitments.}
\end{figure}

The application was made using Haskell, a purely functional programming
language, using nothing but standard libraries, and implementing the necessary
cryptographic operations by hand. Haskell supports arbitrarily large integer
values almost seemlessly which makes creating polynomials of 1024 bits large
integers trivial. To generate the large primes used by the application I used
OpenSSL version 1.1.1.

For the next sections I will cover the theory used to implement the application,
relating it to the code where it makes sense, and discuss the performance of it.

\section{Theory}

The application was made using theory from the Shamir's Secret Sharing
scheme, as well as Feldman's scheme to make it verifiable. 

\subsection{Shamir Secret Sharing Polynomial}

Shamir's Secret Sharing scheme works by creating an n-degree polynomial,
using the secret as the intersect, to ensure that you need $n+1$ points from the
polynomial to reconstruct it. This way all parties can get their own point on
the polynomial as their share, and would have to group up with other parties to
have enough different points to reconstruct. This is an efficient way of
achieving k-out-of-n threshold secret sharing since every party only needs to
know a single point on the polynomial, which has the same size as the secret
itself.

The scheme works over a finite field $F$ from which the secret, $S$, is an
element of. After picking the secret we select $n$ random elements $a_1, a_2,
..., a_n$ in the field to create an n-degree polynomial. Using the secret as
intersect, and the other elements as the other coefficients in the polynomial
$p(x) = S + a_1x + a_2x^2 + ... + a_nx^n$.

I have implemented the scheme over the multiplicative group of integers modulo
q, and the polynomials are represented as lists of integers. So using the secret
as the intersect I create polynomials as seen in figure 3 by selecting $n$
numbers randomly generated in the range between $1$ and $q-1$ where $q$ is a
prime number representing the group.

A point of note here is that the numbers are selected slightly
differenly from how to select the secret, due to how Feldman's scheme works. But
I will come back to this when describing Feldman's scheme.

\begin{figure}[h]
\label{fig:create-polynomial}
\begin{minted}[fontsize=\footnotesize]{haskell}
createPolynomial :: (RandomGen g) => Integer -> Int -> g -> Group -> Polynomial
createPolynomial intersect degree gen q =
    intersect : take degree (randomRs (1, q-1) gen)
\end{minted}
\caption{Code snippet for polynomial creation.}
\end{figure}

\subsection{Party shares}

With a polynomial to represent the secret, the shares given to the parties can
then be calculated by evaluating the polynomial for values $i, i+1, i+2, ...$
and giving party $i$ the share $p(i)$, with $i > 0$. To make sure the shares
don't leak unnecessary information, the parties don't get $p(i)$, but a slighly
modified $p(i)\ mod\ q$ such that the shares themselves are quaranteed to be
part of the finite field. Here it is assumed that party $i$ knows the $i$
corresponding to themself.

\subsection{Reconstructing the Secret}

To reconstruct the secret the parties share their points on the polynomial to
recreate it, usually using Lagrange interpolation since it works in
finite fields.

Lagrange interpolation is calcuated as follows:

\[
L(x) := \sum_{j=0}^k y_j l_j(x)
\]

where $l_j(x)$ is defined as:
\[
l_j(x) := \sum\limits_{\substack{0\leq m \leq k\\m \neq j}}\frac{x - x_m}{x_j - x_m}
\]

and $k+1$ is the number of shares used to reconstruct, $y_j$ is the share at
point $x_j$. 

Since we aren't interested in the whole polynomial, but just the intersect, it
can be modified to be slightly more direct and efficient, by not including the
variable $x$ in the equation, and flipping the values in the denominator.

\[
l_j(x) := \sum\limits_{\substack{0\leq m \leq k\\m \neq j}}\frac{x_m}{x_m - x_j}
\]

The central parts of the calculations are given in figure 4. In the
implementation I have used the Extended GCD algorithm to calculate the inverses
of the elements needed to do the ``division`` in formula. 

\begin{figure}[h]
\label{fig:calculate-lagrange}
\begin{minted}[fontsize=\footnotesize]{haskell}
calcLagrangeTerm :: Group -> Integer -> Integer -> [Integer] -> Integer
calcLagrangeTerm q fxj xj xms = fxj * product ljx
  where ljx = map (\xm -> xm // (xm - xj)) xms
        (//) n m = n * calcInverse m q

reconstruct :: [Share] -> [Int] -> Group -> Integer
reconstruct shares parties q = sum lagrangeTerms `mod` q
  where selectedShares = map (\p -> shares !! (p-1)) parties
        terms = termElems $ map fromIntegral parties  
        lagrangeTerms = 
            map2 (\fxj (xj, xms) -> calcLagrangeTerm q fxj xj xms) selectedShares terms
\end{minted}
\caption{Code snippet for calculating Lagrange Interpolation.}
\end{figure}

\subsection{Verifying the shares}

To make the scheme verifiable for the parties such that a party know their share
is valid, the Shamir Secret Sharing scheme is extended with additional
``commitments`` that the parties receive. This is the difference between Shamir
Secret Sharing scheme and Feldman's scheme. In Feldman's scheme we calculate an
additional $n+1$ values for our $n$-degree polynomial using a prime order
subgroup and a generator from this group. 

In this scheme the coefficients we picked for our polynomial have to be smaller
than the order of the subgroup otherwise the scheme doesn't work. Feldman's
scheme uses the fact that addition in the exponent follows some slightly
different rules, to give an equation where party $i$ can verify that their share
and the commitments match up. 

Since the group has to be prime order and I'm working on the multiplicative
group of integers modulo $q$, the prime $q$ I'm using is a safe-prime, meaning
it has the form $q = 2 \cdot p + 1$ where both $p$ and $q$ are primes. To create
a prime order subgroup from this I take the quadratic residue modulo $q$ of all
elements in the multiplicative group, and this forms the subgroup, which has
order $p$. From this I can pick a generator at random since every element in the
subgroup except $1$ is a generator. In the code this is implemented by selecting
an element uniformly at random from the multiplicative group and calculating the
generator as the quadratic residue modulo $q$ as seen in
figure 5.

\begin{figure}[h]
\label{fig:select-generator}
\begin{minted}[fontsize=\footnotesize]{haskell}
selectGenerator :: (RandomGen g) => g -> Integer -> Integer
selectGenerator gen q = r^2 `mod` q
        where (r, _) = randomR (2, q-1) gen
\end{minted}
\caption{Code snippet for selecting generator for prime order subgroup.}
\end{figure}

In Feldman's scheme we use the generator $g$ and the coefficients from the
polynomial $S, a_1, a_2, ..., a_n$ to calculate the commitments $c_0, c_1, ...,
c_n = g^S, g^{a_1}, g^{a_2}, ..., g^{a_n}$. The commitments are shared to the
parties who can verify among themselves that they have the same commitments,
without revealing their share. To verify their share, a party $i$ can then
calculate $c_0^{i^0}\cdot c_1^{i^1} \cdot c_2^{i^2} \cdot ... \cdot c_n^{i^n}$
and compare it to $g^{p(i)}$ where $p(i)$ is their share. This works since:

$g^{p(i)} = g^{S + a_1\cdot i + a_2 \cdot i^2 + ... + a_n \cdot i^n} = 
g^{S}\cdot g^{a_1 \cdot {i^1}} \cdot g^{a_2 \cdot {i^2}} \cdot ... \cdot g^{a_n
\cdot {i^n}} = $

$
g^{S^{i^0}}\cdot g^{a_1^{i^1}} \cdot g^{a_2^{i^2}} \cdot ... \cdot g^{a_n^{i^n}} = 
c_0^{i^0}\cdot c_1^{i^1} \cdot c_2^{i^2} \cdot ... \cdot c_n^{i^n}$

The reason we use two different primes is that while omitted in the calculations
for clarity, everything is calculated in the subgroup generated by the generator
$g$, so in reality it is $g^{p(i)\ mod\ p}\ mod\ q$ since the generator is used
to generator a group of prime order $p$, we know that $g^p = 1$ which implies
$g^a = g^{a\ mod\ p}$.

As is, this scheme would leak information on the secret $S$ so a minor
modification of the scheme is usually done to prevent this, but this was beyond
the scope of this project.

Since this scheme uses a lot of exponentiation I have implemented a function for
faster modular exponentiation. Without using this, relatively small calculations
like $45384516^{1464645052}$ wouldn't even complete before the process was
killed.

From this I have implemented the calculations as seen in figure 6.

\begin{figure}[h]
\label{fig:calculate-feldman}
\begin{minted}[fontsize=\footnotesize]{haskell}
calculateFeldmanProduct :: [Integer] -> Generator -> Integer -> Group -> Integer
calculateFeldmanProduct commitments g i q = product vals `mod` q
    where vals = map2 (\c p -> powerMod c p q) commitments commitPowers
          commitPowers = map (i^) $ [0..]
\end{minted}
\caption{Code snippet for calculating Feldman's product.}
\end{figure}


\subsection{Primes}

The implementation uses more than a single prime. Depending on the secret the
smallest working prime is picked for demonstrational purposes. The primes used
are of 8, 16, 32, 64, 128, 256, 512, and 1024 bits of length, and generated
using the command ``openssl prime -generate -safe -bits [size]''

As a consequence of using hard-coded primes the application doesn't work for
secrets bigger than the largest prime, but for the purposes of this application
it should be more than necessary. 

For the sake of completeness the primes are included in figure 7.
\begin{figure}[ht]
\label{fig:primes}
\center
\begin{tabular}{ l l }
Bits & Prime \\
8    & 227 \\
16   & 51407 \\
32   & 3260360843 \\
64   & 16855205958470386187\\
128  & 336809720957688272968116945989114351147 \\
256  & 10093244343192998281619872079169332717505060290270351828012\\
     & 8748060882240133359 \\
512  & 12754709477010474220782419068422931704491832184660376415376\\
     & 49129662835681510280325117671740229982353299299584844280080\\
     & 7715447900267393272594855354516567627 \\
1024 & 17515023902464469416169064215919178191091046333390979796239\\
     & 96411136449740117840979007830349408818890443448677163307017\\
     & 16017301210554996353255650234107823397774674975724010953046\\
     & 91178561758182117442207104772836842423542561213945128553213\\
     & 36806373141154311947173292682082569803172939028042913706998\\
     & 09368006311239
\end{tabular}
\caption{Safe prime numbers used by the application to perform the scheme.}
\end{figure}

\section{Performance}

While good performance wasn't a goal of this project, and there are no messages
being transmitted over the network, it is still interesting to see how fast it
is to reconstruct and verify different configurations of the protocol.

For the tests I will be running all the different prime numbers, for groups of
increasing size 5, 10, 50, 100, 500, and 1000. The tests will be split into 2
groups, 1 for reconstructing the secret, the other for a single party verifying
their secret. To time the tests I will user the unix time command and take the
user time. In total there are 96 tests split between the two groups. No tests
are repeated and no averages, median, standard deviation, etc. is used. This is
simply to give a rough idea of the time to compute the schemes.

One minor thing to note is that the tests for 500 and 1000 parties for the 8 bit
prime doesn't return the correct result since there are more parties than the
prime can contain. The time is still included since the calculations are still
the same, even though they result in the wrong secret being reconstructed.

\subsection{Reconstruct}

The results of the tests are given in figure 8. Reconstructing the secret is
reasonably fast even for the largest test group. The biggest contributing factor
to the time to reconstruct seems to be the number of parties involved. With the
time to reconstruct even a 1024 secret being negligible. For the numbers which
don't fit inside a standard 64 bit integer the time also seems to roughly double
as the bitsize doubles. 

\begin{figure}[h]
\label{fig:reconstruct-test-results}
\center
\textbf{Prime size}\\[0.5em]
\textbf{Parties}
\begin{tabular}{l|cccccccc}
     & 8    & 16    & 32    & 64    & 128   & 256   & 512   & 1024  \\
\hline
5    & 5    &  1    &  1    &  1    &  1    & 1     & 0     & 0     \\
10   & 6    &  1    &  0    &  1    &  1    & 1     & 1     & 2     \\
50   & 22   &  3    &  5    &  6    &  7    & 7     & 7     & 20    \\
100  & 24   &  21   &  21   &  22   &  23   & 33    & 56    & 126   \\
500  & 708  &  844  &  934  &  931  &  1222 & 2040  & 4778  & 13583 \\
1000 & 4197 &  5603 &  4923 &  5434 &  7752 & 14143 & 36140 & 106197\\
\end{tabular}
\caption{Performance tests for reconstructing the shared secret using 
         Lagrange Interpolation. The results are given in milliseconds}
\end{figure}

\subsection{Verify}

The results of the tests are given in figure 9.
Since I am only testing the time for a single party to verify their share the
party picked is the party $\lfloor party / 2\rfloor$. 

The results show that verifying the share is faster than reconstructing the
secret for the most part.

\begin{figure}
\label{fig:verify-test-results}
\center
\textbf{Prime size}\\[0.5em]
\textbf{Parties}
\begin{tabular}{l|cccccccc}
     & 8    & 16    & 32    & 64    & 128   & 256   & 512   & 1024  \\
\hline
5    & 0    &  1    & 1     & 1     & 0     & 2     & 3    & 8      \\
10   & 5    &  1    & 1     & 1     & 10    & 2     & 5    & 15     \\
50   & 7    &  3    & 0     & 0     & 5     & 9     & 16   & 58     \\
100  & 14   &  8    & 9     & 12    & 15    & 17    & 49   & 152    \\
500  & 465  &  450  & 467   & 591   & 634   & 765   & 1069 & 2034   \\
1000 & 3375 &  3347 & 3370  & 3990  & 4121  & 4618  & 5631 & 8850   \\
\end{tabular}
\caption{Performance test for verifying a single party's share using Feldman's
         scheme. The results are given in milliseconds.}
\end{figure}

\section{Code}

Here is included the full source code of the project.

\inputminted{haskell}{../interactive-ss.hs}

\end{document} 
